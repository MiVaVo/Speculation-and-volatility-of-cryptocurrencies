\HeaderA{garch-methods}{Methods for Fitted GARCH Models}{garch.Rdash.methods}
\aliasA{coef.garch}{garch-methods}{coef.garch}
\aliasA{fitted.garch}{garch-methods}{fitted.garch}
\aliasA{logLik.garch}{garch-methods}{logLik.garch}
\aliasA{plot.garch}{garch-methods}{plot.garch}
\aliasA{predict.garch}{garch-methods}{predict.garch}
\aliasA{print.garch}{garch-methods}{print.garch}
\aliasA{residuals.garch}{garch-methods}{residuals.garch}
\keyword{models}{garch-methods}
\keyword{ts}{garch-methods}
\begin{Description}\relax
Methods for fitted GARCH model objects.
\end{Description}
\begin{Usage}
\begin{verbatim}
## S3 method for class 'garch':
predict(object, newdata, genuine = FALSE, ...)
## S3 method for class 'garch':
coef(object, ...)
## S3 method for class 'garch':
residuals(object, ...)
## S3 method for class 'garch':
fitted(object, ...)
## S3 method for class 'garch':
print(x, digits = max(3, getOption("digits") - 3), ...)
## S3 method for class 'garch':
plot(x, ask = interactive(), ...)
## S3 method for class 'garch':
logLik(object, ...)
\end{verbatim}
\end{Usage}
\begin{Arguments}
\begin{ldescription}
\item[\code{object, x}] an object of class \code{"garch"}; usually, a result
of a call to \code{\LinkA{garch}{garch}}.
\item[\code{newdata}] a numeric vector or time series to compute GARCH
predictons.  Defaults to \code{eval(parse(text=object\$series))}.
\item[\code{genuine}] a logical indicating whether a genuine prediction
should be made, i.e., a prediction for which there is no target
observation available.
\item[\code{digits}] see \code{\LinkA{printCoefmat}{printCoefmat}}.
\item[\code{ask}] Should the \code{plot} method work interactively?  See
\code{\LinkA{interactive}{interactive}}.
\item[\code{...}] further arguments passed to or from other methods.
\end{ldescription}
\end{Arguments}
\begin{Details}\relax
\code{predict} returns +/- the conditional standard deviation
predictions from a fitted GARCH model.

\code{coef} returns the coefficient estimates.

\code{residuals} returns the GARCH residuals, i.e., the time series
used to fit the model divided by the computed conditional standard
deviation predictions for this series. Under the assumption of
conditional normality the residual series should be i.i.d. standard
normal.  

\code{fitted} returns +/- the conditional standard deviation
predictions for the series which has been used to fit the model.

\code{plot} graphically investigates normality and remaining ARCH
effects for the residuals.

\code{logLik} returns the log-likelihood value of the GARCH(p, q)
model represented by \code{object} evaluated at the estimated
coefficients. It is assumed that first max(p, q) values are fixed.
\end{Details}
\begin{Value}
For \code{predict} a bivariate time series (two-column matrix) of
predictions. 

For \code{coef}, a numeric vector, for \code{residuals} and
\code{fitted} a univariate (vector) and a bivariate time series
(two-column matrix), respectively.

For \code{plot} and \code{print}, the fitted GARCH model object.
\end{Value}
\begin{Author}\relax
A. Trapletti
\end{Author}

